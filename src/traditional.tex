The hidden Markov development model is compared to
a more traditional two-step modelling approach.
Denote $\tau \in \{2, ..., M\}$ the final body
training data point, and
$\bm{\rho} = (\rho_{1}, \rho_{2}) \in \{2, ..., M\}$,
where $\rho_{1} < \rho_{2}$,
a vector of tail start and end training window
development points, respectively.
While these constants could in theory vary over
experience periods, there is typically insufficient
data to do so.
In the traditional approach, the chain ladder
method is first fit to training data up
to development period $\tau$ inclusive and
predictions of the lower diagonal loss
triangle are made up to and including $\tau$, only.
Secondly, the tail model is fit to data
lying within the development period
interval $[\rho_{1}, \rho_{2}]$,
and predictions from $\tau$ made to 
some arbitrary development lag, $j^{*}$.
The challenge and art of this two-step process
is in finding a value for $\tau + 1$ that
identifies the development lag where
losses are plateauing in the tail,
and finding values for $\rho_{1,2}$ that
identify a suitable decaying curve of link ratios.
While $\tau = \rho_{1}$ in some cases,
more generally $\rho_{1} \leq \tau$.
Note, this presents a difference between
the two-step and hidden Markov model approaches:
the hidden Markov model identifies clear
body-to-tail switch-over points,
whereas the the interval of data
delineated by
$\rho_{1,2}$ in the two-step process
might overlap
the final body training point, $\tau$.

To maintain comparability to the hidden
Markov model above, the two-step
approach is also implemented in Stan
and both models estimated
with a shared variance. In real
applications, uncertainty 
from body to tail models is typically
ignored, but this may
unfairly penalise the two-step process
compared to the hidden Markov models. The
two-step approach differs from
equation \ref{eq:hmm} in only two
ways:

\begin{align}
\begin{split}
	y_{ij} &\sim 
	\begin{cases}
		\mathrm{Lognormal}(\mu_{1}, \sigma_{ij}) \quad j \leq \tau\\
		\mathrm{Lognormal}(\mu_{2}, \sigma_{ij}) \quad \rho_{1} \leq j \leq \rho_{2}
	\end{cases}\\
	\log \bm{\alpha}_{1:\tau - 1} &\sim \mathrm{Normal}(0, 1)\\
\end{split}
\end{align}

where now the decision between the two
models is decided by $\tau$ and $\rho_{1,2}$.
The further exception in the two-step approach
is the use of a standard normal prior
on the log-scale link ratios, rather than the
constrained prior used in the hidden Markov models.
The two-step model's $\tau - 1$ link
ratios are all estimated directly and do not suffer
from non-identifiability.

This paper has proposed
a hidden Markov loss development model
for insurance loss triangles that combines
body and tail development models,
and automates the selection of body-to-tail
switch-over points.
Simulation-based calibration validated
the hidden Markov model implementation
as being unbiased, and across
a range of different datasets,
the hidden Markov model variants
provided similar results to, 
and often out-performed,
the traditional two-step approach
and a latent change-point model.
The hidden Markov models'
automated detection
of body and tail processes
more gracefully captures
loss development dynamics
that may vary over triangle
experience periods, as well as
reduce 
analysts' degrees of freedom
that make the traditional
two-step approach reliant
on difficult-to-reproduce
and variable subjective
decisions.

The hidden Markov development model 
posits a clear data-generating process for
loss development dynamics.
Although referred to here as `body' and `tail',
these two latent states might equivalently
be thought of as flexible and smooth
periods of loss development, and can interchange
depending on the context, as in the 
HMM-$\nu$ model variant here.
In this way, the hidden Markov model is not a strictly
analagous implementation of the two-step
approach,
as the two-step approach allows analysts
to choose tail model training data windows
that overlap the body-to-tail cutoff point.
Thus, the same data points may be used in
estimation of body and tail processes, rather
than a discrete cutoff point between the two.
Despite this flexibility, the two-step approach
is not a single generative model, and should an
analyst choose values for $\tau$ and $\rho_{1,2}$
that are not representative of a particular
experience period, the predictions from
such a model could be extremely biased,
as shown in the example of Figure \ref{fig:numerical}.
While the hidden Markov model may still
make relatively poor predictions for those experience
periods with little data (e.g. see the last
column of panel A in Figure \ref{fig:numerical}),
uncertainty in the true latent state, $\bm{z}$,
is more-accurately accounted for.

Although most of the hidden Markov model variants
performed consistently better on average
than the two-step
approach on the curated industry datasets
of \cite{meyers2015}, the approaches were
more similar on the five literature triangles.
These two sets of data present different
case studies. The industry triangles
have been selected to encompass relatively
large insurers with mostly stable
loss dynamics \citep[see][appendix A]{meyers2015}
over a period of 10 years.
Due to the number of triangles, the two-step
approach's cutoff points, $\tau$ and $\rho_{1,2}$,
were chosen based on average empirical link
ratios, which might not have been the best
selection points for some triangles.
By contrast, the literature triangles encompass
more accident periods per triangle but also smaller
books of business \citep[e.g. the medium-sized
triangles from][]{balona2022}, and
more variability in the tail than 
present in the industry triangles.
Previous papers on loss development models
combining body and tail dynamics
have not considered the breadth of triangles
and lines of business used here. For instance,
\cite{englandverrall2001}, \cite{verrall2012},
and \cite{verrall2015} all
used a single triangle to illustrate their approaches,
and 
\cite{zhang2012} used a dataset of
10 workers' compensation triangles and
did not consider other lines of business or
more volatile triangles. Moreover,
the previous papers did not compare
their approaches to the more common
two-step approach applied in actuarial
practice.
The datasets used here are provided alongside
this article in the repository
for ease of access
and comparison of other loss development
modelling approaches.

Both the hidden Markov and two-step
approaches demonstrated relatively poor
calibration on the out-of-sample
data from the \cite{meyers2015}
dataset (Figure \ref{fig:percentiles}). 
Primarily, 
the out-of-sample
predictions were often too
uncertain, producing a
predominance of percentiles falling
within the central range of
possible percentiles.
Few articles have shown calibration
plots from fully-Bayesian posterior
distributions on out-of-sample
loss development data, so this pattern of calibration requires
further inspection in the literature.
\cite{meyers2015} reports on 
calibration using the same data set,
showing relatively well-calibrated
predictions. However, we note
that \cite{meyers2015} calculated
the percentile of the total ultimate
losses in each triangle on a lognormal
distribution with mean and variance
informed by the total ultimate losses
from their models. Thus, these
are not directly comparable because
the results above average percentiles
over each posterior predictive distribution
in the test data, not a distribution informed
by just the average of posterior predictive
distributions.

The hidden Markov models presented here
could be extended in a number of useful
ways. Notably, the transition matrix
probabilities might be parametric
or non-parametric functions of covariates,
such as premium volume in each experience
period or inflation levels in each
calendar period, or include
hierarchical effects for experience
and development periods.
The Bayesian framework, alongside the
hidden Markov models implemented
here and available in the supplementary
material, make these extensions
highly accessible.
Additionally, the hidden Markov
model framework is general enough
to include any body or tail
model, not just the chain ladder
and generalised Bondy forms.
For instance, there are a
number of inverse power curves
to use for tail modelling
\citep{tailfactors2013,evans2015,clark2017},
and extensions and variations on the chain ladder
model have been commonplace \citep{englandverrall2002}.
Although the analyses in this paper focused on
paid losses, the same models could be applied
to estimates of reported losses (i.e.
paid loss plus estimates of reserve),
or joint modelling of both paid and reported
losses \citep[e.g. see][for one approach]{zhang2010}.

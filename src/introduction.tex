Actuarial science is one of the oldest areas
of applied statistics, 
having been founded in the seventeenth
century, and insurance continues to pose
many interesting statistical questions
about risk and uncertainty.
A core task in actuarial science is
\textit{loss development} or \textit{loss reserving}.
Loss development is the practice of
predicting the total 
payments due on all 
claims expected from
a set of insurance policies \citep{englandverrall2002,zhang2012}.
Thus, loss development requires
forecasting not only future payments for
claims already reported to insurance companies 
but payments for claims yet to be reported.
In this way, insurance is unique in that 
the eventual cost of 
an insurance policy to the insurance
company is unknown at the time of sale
because insurance policies cover, for the most part,
prospective accidents only.
Together with the time it takes to report claims
to insurance companies, and the time it can
take for an insurance company to arrive
at a final claim loss payment amount,
these features of insurance make knowing
the total losses an insurance company will
be liable to pay significantly uncertain. 

The total losses due for claims
from a pool of insurance policies
are typically referred to as
the \textit{ultimate} losses, and are composed
of the loss payments already made on opened
claims (paid losses), plus an amount estimated by actuaries
for payments on claims yet to be made (loss reserves).
This latter amount is frequently referred to as
\textit{incurred but not reported} (IBNR), and
includes future payments on
claims already reported (`incurred by not enough reported'),
as well as payments for claims not yet reported (`incurred by not
yet reported').
In non-life insurance, 
such as property and casualty insurance,
IBNR can be substantial as it may
take years for all claims on a set of policies to be reported and settled.
Therefore, the accurate estimation of IBNR is key to evaluating an insurance company's
performance and solvency \citep{beard1960,bornhuetter1972,friedland2010,englandverrall2002,wuthrich2008}.
Transparent assessment of the uncertainty in IBNR 
is also critical to the legal responsibilities
of insurers and reinsurers, such as for the Solvency II directive
\citep{england2019,frohlich2018,munroe2018}.

Estimation of loss reserves 
has received considerable attention from actuaries, econometricians,
and statisticians for decades 
\citep[e.g. for some key developments, see][]{bornhuetter1972,clarke1974,
taylor1977,taylor1983,mack1993,barnett2000,englandverrall2001,englandverrall2002,
taylor2003,wuthrich2008}. 
Loss development models are typically applied to sets of aggregated and tabulated
data known as \textit{loss triangles}, where the rows represent
intervals of time when accidents occurred, known as experience periods,
and the columns represent evaluation points, known as development periods,
of the cumulative or incremental cost of those claims reported in each accident year
(see Figure \ref{fig:schematic} for a worked example).
The data structure represents an upper triangular matrix, hence
the name loss triangle, and it is the task of loss development modelling 
to complete, and potentially extend, the
lower diagonal to produce predictions of ultimate loss.
From early deterministic and 
algebraic approaches \citep[e.g.][]{scurfield1968,
bornhuetter1972,clarke1974,taylor1977}, loss development modelling has
advanced to utilise a wide, and growing, variety of statistical methods, 
including parametric and non-parametric regressions \citep{mack1994,englandverrall2001,
englandverrall2002,lally2018}, Bayesian estimation \citep{englandverrall2002,dealba2002,
zhang2012,meyers2015}, differential equations \citep{gesmann2020}, and 
neural networks and machine learning methods 
\citep{kunce2017,kuo2019,almudafer2022}.

A key inferential quantity from any loss development
model is the set of multiplicative factors transforming 
the losses at development period one to
the ultimate losses at development period $\infty$,
known as the \textit{loss development factors} or \textit{link ratios}. 
While many models include link ratios as an explicit parameter to be 
inferred, notably the family of chain ladder methods
\citep[e.g.][]{mack1993,englandverrall2002},
others derive link ratios as a generated quantity \citep{englandverrall2001,meyers2015}. 
Ideally, link ratios for cumulative losses will smoothly decline over time towards unity,
indicating that the losses are approaching their ultimate value at later development
periods.
In practice, however, link ratios often include periods of volatility, particularly
for early development periods, and may further encode systematic
and non-systematic effects across experience periods or 
development period (e.g. the influence of the Covid-19 pandemic).
Depending on the degree of volatility and the amount of data available,
link ratios estimated using popular methodologies
may be insufficient to infer the ultimate losses for each experience period.
For instance, at the latest development period in the data, link ratios
may not be nearing unity.
Consequently, estimation of ultimate losses will require extending the link ratios
to outside the domains of the focal triangle to include additional
link ratios, known as \textit{tail factors}, that make assumptions
about how the losses will continue to develop in the future.

Like the loss development literature discussed above, 
tail factor estimation has had its own expansive history
of deterministic and stochastic methods \citep{tailfactors2013}, and is of particular
importance in `long-tailed' lines of business, such as workers' compensation or general
liability, where experience periods might display continued growth and 
volatility of loss amounts at relatively late development periods.
Of the various approaches used to calculate tail factors, many 
use a second model fit to a portion
of the focal triangle or to a subset of the link ratios
that conveys how the triangle may behave
in the tail, close to ultimate loss \citep{tailfactors2013}.
These models typically 
infer a parametric, monotonically
increasing growth curve of losses from the training data, 
such as various forms of
inverse power curves \citep[e.g.][]{sherman1984,evans2015,clark2017}.
The link ratios derived from this tail model are then appended to
the link ratios estimated from the primary loss development model
to produce predictions an
arbitrary number of development lags into the future.
For clarity, the primary link ratios will be referred to
in this paper as the `body' link ratios to distinguish them from 
link ratios estimated in the tail.

The two-step process of estimating body and tail link ratios
includes a number of 
subjective decisions that may not be optimal. 
Namely, it necessitates manually selecting a development 
period where the tail link ratios should take over from
the body link ratios, and manually selecting portions of
loss triangles to be used as training data that best match
body and tail development model assumptions.
The choice of development periods to use in each case
can have important implications for predictions of ultimate
loss.
For instance, including periods
of non-monotonic growth into tail model training
data would bias
tail factors, and ultimate losses, unreasonably high. 
Similarly, predicting
ultimate losses using tail factors applied too early
in a triangle
could miss important development features at later
development lags and bias ultimate losses too low. 
Moreover, the manual selection of training data windows
and body-to-tail cut-off points
are difficult to reproduce and
opens analysts to many `researchers degrees of freedom' \citep{simmons2011}.

Methods that simultaneously estimate body
and tail link ratios 
present an attractive alternative
to traditional two-step processes. However, only a few
solutions have been proposed.
\cite{englandverrall2001} presented a generalised
additive model for smooth estimation of loss development
factors that can be extrapolated to points futher
than the existing data. Although flexible and able to
integrate different functional forms and covariates
of loss development processes, this approach
still necessitates careful selection of training data
to include in the model so that regions of volatility
do not bias loss development curves. Generalised additive models
further require selecting the family of splines and
number of knots to apply, which may lead to another
set of decisions analysts must act on.
\cite{zhang2012}, alternatively,
implemented a hierarchical Bayesian logistic growth curve model to
cumulative loss data. However, fitting a single parametric
curve that assumes monotonicity in expectation might under-estimate
systematic volatility in portions of the triangle that
analysts do not want to label as residual noise.
Finally, \cite{verrall2012} use reversible jump Markov
chain Monte Carlo to combine a Bayesian chain ladder model,
applied before some tail cut-off point, with an exponential
decay process after the cut-off, and allow the model to infer
where the cut-off should occur. Additionally, 
\cite{verrall2015} demonstrate how the same model
can be estimated in a Bayesian model averaging context.
Although this approach is arguably the most flexible,
it requires either bespoke sampling algorithms (i.e. reversible
jump Monte Carlo) or
multiple model fits (for model averaging purposes),
which may dissuade analysts and researchers.

This paper proposes the use of hidden Markov models
to simultaneously estimate body and tail link ratios
in a single model from a variety of loss triangles. 
Hidden Markov models are 
primarily discrete mixture models postulating an unknown 
latent state
underlying and generating patterns of observed data,
and have found a multitude of applications, 
from speech recognition \citep[e.g.][]{rabiner1989}
to animal behaviour \citep[e.g.][]{leos2017}.
Indeed, Markov processes have been previously used
in micro-level claim 
modelling \citep[e.g.][]{hesselager1994} but
have not been applied to aggregate insurance
loss triangles.
As described in this paper, hidden Markov 
development models decompose 
a loss triangle into a sequence of body and tail
processes.
Hidden Markov models are easily fit
in existing, open-source software, and can cater
for complex data-generating assumptions,
such as understanding the impact of covariates
or including non-parametric patterns of loss development 
\citep{englandverrall2001},

Below, the hidden Markov model is formulated
and validated with simulated examples and 
subsequently compared
to a traditional two-step process using
a number of data sets. The hidden Markov
models are fit using Bayesian estimation
in Stan \citep{stan2017} following a modern Bayesian
workflow \citep{gelman2020}.
All Python and Stan code, and all datasets, to 
reproduce the results
are accessible at the Github repository
\url{https://github.com/LedgerInvesting/hidden-markov-development-2024}.

